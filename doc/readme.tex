% Created 2018-05-21 Mon 22:22
% Intended LaTeX compiler: pdflatex
\documentclass[11pt]{article}
\usepackage[utf8]{inputenc}
\usepackage[T1]{fontenc}
\usepackage{graphicx}
\usepackage{grffile}
\usepackage{longtable}
\usepackage{wrapfig}
\usepackage{rotating}
\usepackage[normalem]{ulem}
\usepackage{amsmath}
\usepackage{textcomp}
\usepackage{amssymb}
\usepackage{capt-of}
\usepackage{hyperref}
\usepackage[newfloat]{minted}
\usepackage{xeCJK}
\setCJKmainfont{Songti SC}
\usepackage{latexsym}
\usepackage[mathscr]{eucal}
\usepackage[section]{placeins}
\usepackage{float}
\usepackage{svg}
\author{计72 陈嘉杰}
\date{\today}
\title{OOP 个人大作业 Problem 4 Exact One-pass Synthesis of Digital Microfluidic Biochips 论文实现}
\hypersetup{
 pdfauthor={计72 陈嘉杰},
 pdftitle={OOP 个人大作业 Problem 4 Exact One-pass Synthesis of Digital Microfluidic Biochips 论文实现},
 pdfkeywords={},
 pdfsubject={},
 pdfcreator={Emacs 26.1 (Org mode 9.1.13)}, 
 pdflang={English}}
\begin{document}

\maketitle
\tableofcontents

\section{任务说明}
\label{sec:org7890733}
这篇论文的目的是,将 DMFB (Digital Microfluidic Biochps)的合成路径问题转化为 SAT 问题,再采用 SAT Solver (本文中采用的是 z3 prover)解决。论文中针对不同类型的要求,添加了若干的变量和它们之间的约束关系,于是根据求得的结果可以反推得到实际的合成过程。

\section{任务实现}
\label{sec:org866e4be}
\begin{enumerate}
\item 使用 CMake 构建系统,把 z3 加入到依赖之中。
\item 从 github.com/UCRMicrofludics/MFSimStatic 获取了一些测试数据,放在 testcase/Assays/ 下。
\item 采用了上述测试数据的文件格式,编写了 \texttt{\{Node,Graph\}.\{h,cpp\}} 进行流程图的读入和解析,并且可以输出到 dot 文件再通过 graphviz 打印出 png 图片。
\item 在 \texttt{Solver.\{h,cpp\}} 中实现了主要的功能,包括:创建相关的变量,然后根据不同类型的约束条件,在不同的函数里添加到 SAT Solver 中,然后要求 SAT Solver 输出结点总数最少的结果,最后从结果中还原出合成的完整过程,然后输出到文件并合成为 gif 动画。
\item 支持的结点类型:DISPENSE, MIX, DETECT, OUTPUT 。由于 SPLIT 和 DILUTE 在论文中没有相关的约束条件,不予实现。
\item 实现了 fluidic constraint ,方法是如果两个液滴接近,那么液滴将会被合并。
\item 测试案例在目录 testcase/ 下,部分的运行结果放在了 solutions/ 的相应子目录下,可供查看运行结果。
\end{enumerate}

\section{结点编号约定}
\label{sec:org4e918c1}
\begin{enumerate}
\item 输入格式与 MFSimStatic 相同,每个结点的编号从 1 连续增大。
\item 显示在图片和动画中的数字为下标,即对应结点的编号减一。
\item 合并操作、检测操作都会产生新的一个液滴,液滴对应的编号为这个操作自己的编号。
\end{enumerate}

\section{程序编译环境}
\label{sec:orgc29ee50}
\begin{enumerate}
\item 操作系统: macOS
\item 编译器: LLVM/Clang 6.0.0
\end{enumerate}

\section{遇到的问题和解决方案}
\label{sec:org895ad27}
\begin{enumerate}
\item 输入的数据中未指明 mixing 所需要的区域的大小,目前是以 2x2 写在代码之中。
\item 论文提供的 SAT 约束条件并不能保证不必要的液滴的出现和移动,这一点我采用让 SAT Solver 对结点总数最少进行优化得以解决。
\item 使用 z3 过程中遇到了它代码中的 BUG ,已提交到上游并且在 master 分支已经修复,在我的代码中则选择绕过了它。
\end{enumerate}
\end{document}